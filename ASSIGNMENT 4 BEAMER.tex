%%%%%%%%%%%%%%%%%%%%%%%%%%%%%%%%%%%%%%%%%%%%%%%%%%%%%%%%%%%%%%%
%	
% Welcome to Overleaf --- just edit your LaTeX on the left,
% and we'll compile it for you on the right. If you open the
% 'Share' menu, you can invite other users to edit at the same
% time. See www.overleaf.com/learn for more info. Enjoy!
%
%%%%%%%%%%%%%%%%%%%%%%%%%%%%%%%%%%%%%%%%%%%%%%%%%%%%%%%%%%%%%%%

% Inbuilt themes in beamer
\documentclass{beamer}

%packages:
% \usepackage{tfrupee}
% \usepackage{amsmath}
% \usepackage{amssymb}
% \usepackage{gensymb}
% \usepackage{txfonts}

% \def\inputGnumericTable{}

% \usepackage[latin1]{inputenc}                                 
% \usepackage{color}                                            
% \usepackage{array}                                            
% \usepackage{longtable}                                        
% \usepackage{calc}                                             
% \usepackage{multirow}                                         
% \usepackage{hhline}                                           
% \usepackage{ifthen}
% \usepackage{caption} 
% \captionsetup[table]{skip=3pt}  
% \providecommand{\pr}[1]{\ensuremath{\Pr\left(#1\right)}}
% \providecommand{\cbrak}[1]{\ensuremath{\left\{#1\right\}}}
% %\renewcommand{\thefigure}{\arabic{table}}
% \renewcommand{\thetable}{\arabic{table}}      

\setbeamertemplate{caption}[numbered]{}

\usepackage{enumitem}
\usepackage{tfrupee}
\usepackage{amsmath}
\usepackage{amssymb}
\usepackage{gensymb}
\usepackage{graphicx}
\usepackage{txfonts}

\def\inputGnumericTable{}

\usepackage[latin1]{inputenc}                                 
\usepackage{color}    
\usepackage{textcomp, gensymb}         
\usepackage{array}                                            
\usepackage{longtable}                                        
\usepackage{calc}                                             
\usepackage{multirow}                                         
\usepackage{hhline}                             
\usepackage{mathtools}
\usepackage{ifthen}
\usepackage{caption} 
\providecommand{\pr}[1]{\ensuremath{\Pr\left(#1\right)}}
\providecommand{\cbrak}[1]{\ensuremath{\left\{#1\right\}}}
\renewcommand{\thefigure}{\arabic{table}}
\renewcommand{\thetable}{\arabic{table}}   
\providecommand{\brak}[1]{\ensuremath{\left(#1\right)}}

% Theme choice:
\usetheme{CambridgeUS}

% Title page details: 
\title{AI1110 \\ Assignment-4} 
\author{Pettugadi Pranav CS21BTECH11063}
\date{\today}
\logo{\large \LaTeX{}}


\begin{document}

% Title page frame
\begin{frame}
    \titlepage 
\end{frame}
\logo{}


% Outline frame
\begin{frame}{Outline}
    \tableofcontents
\end{frame}



\section{Question}
\begin{frame}{Question}
    \begin{block}{\textbf{Papoullis 4-11:} } 
       The space S contains of all points $t_i$ in the interval (0,1) and $P\{0 \le t_i \le y\}=y$ 
       for every$y \le 1$ . The function G(x) is increasing from $G(-\infty)=0$ to $G(\infty)=1$
; hence it has an inverse $G^{-1}(y) = H(y)$. The random variable$ X(t_i)= H(t_i)$. Show that $F_X(x)=G(x)$     \end{block}
     
\end{frame}



\section{Solution}
\begin{frame}{Solution}
  \begin{block}{}
      From the given information;\\ t_i(0,1)\\
       $P\{0\le t_i \le y\}=y$\\
        $X(t_i) =H(t_i)$ \\ $H(x) = G^{-1}(x)$\\
        \end{block}
        
        \begin{align}
            F_X(x) &= P\{X(t_i) \le x\}\\
                   &=  P\{H(t_i) \le x\}\\
        \end{align}
        
        \end{frame}
        
        \begin{frame}
        \begin{align}
            X(t_i) \le x\\
            H(t_i) \le x\\
            t_i \le H^{-1}(x)\\
            t_i \le G(x)\\
            \end{align}
            \begin{align}
            P\{X(t_i) \le x\} = P\{t_i \le G(x)\}\\
            \end{align}
            \end{frame}
            
            \begin{frame}
            As $0 \le G(x) \le 1$
            
            \begin{align}
            P\{t_i \le G(x)\} &= P\{0 \le t_i \le G(x)\}\\
            &= G(x)\\
            \end{align}
        $P\{X(t_i) \le x\}=G(x)$  \because  eq (12)\\
 \therefore $F_X(x)=G(x)$
  \end{frame}
  
 
\end{document}